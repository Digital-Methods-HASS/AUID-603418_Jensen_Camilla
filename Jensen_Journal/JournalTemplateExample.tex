\documentclass{article}
\usepackage[utf8]{inputenc}

\title{Digital Methods: Learning Journal Template}
\author{Camilla Kremmer Jensen}
\date{Autumn 2019}

\begin{document}

\maketitle

\section{Today's Date}
\subsection{Thoughts / Intentions}
\subsection{Action}
\subsection{Results}
\subsection{Final Thoughts}

\pagebreak{}

\section{06/11/2019}
\subsection{Thoughts / Intentions}

\textbf{08:44am}: So I just read the homework for tomorrows lesson. I feel a bit lost already, but I'll try to make it work. 
\textbf{09:37am}: I will try to clean those messy dates 
I already created a Github user account and the Overleaf account 




\subsection{Action}

\begin{itemize}
    \item I already created a Github user account. 
    \item I already created a Overleaf account too. 
\end{itemize}

Creating a repository on Github:

\begin{itemize}
    \item Click on profile icon in top right corner and click ‘your repositories’
    \item Click ‘new’ and entering details - named repository ‘learning-journal’
    \item Selected initialize this repository with a README and created repository
    \item Uploading overleaf weekly template sample
    \item Commited file with description
\end{itemize}
 
Overleaf:
\begin{itemize}
    \item Used sample template in Overleaf and edited title/abstract 
    \item 
    \item Attempt to recompile presents \textbf{ERROR} - still recompiles to the weekly journal template. 
    \item Deleted template and recompile section dissapeared with \textbf{ERROR} - “Unknown main document. Please choose the main file for this project in the project menu”
\end{itemize}

Data Carpentry Exercises:

\textit{Messy Spreadsheet}

\begin{itemize}
    \item Opened data sets from previously downloaded excel spreadsheets
    \item Found ‘tabs’ in excel in bottom left corner
    \item Discussing messy data
    \item Color used for indication
    \item Multiple variables in column and row
    \item Multiple values in single cells
    \item Inconsistencies in null, 0 and false in each tab
    \item Inconsistencies and use of asterisk
    \item Blank cells are not adequate indicators
    \item Multiple tabs in one spreadsheet
\end{itemize}

\subsection{Results}

\begin{itemize}
    \item Github - 
    \item Overleaf - . 
\end{itemize}

\subsection{Final Thoughts}
\begin{itemize}
\textbf{2:03pm}: Im still a bit confused about it all. But I think I will get it at some point. I feel a lot more confident with overleaf. I will continue with the exercise tomorrow morning.  
\end{itemize}\bigskip


\section{20/11/2019}
\subsection{Thoughts/Intentions}

\begin{itemize}
\item \textbf{08:50am}: So today I found my final project on "Our World in Data" 
\end{itemize}

\subsection{Action}

"Our World in Data" website 

\begin{itemize}
    \item I went to the website 
    \item Had a little look around 
    \item Clicked on the category "Global Health"  
    \item Looked at some of the subsections  
    \item I founded the category "Maternal mortality"
    \item Then I downloaded the csv-file and had a little look on the data. 
    \item And decided that this would be my project for the exam 
\end{itemize}

\subsection{Final Thoughts}
\textbf{09:20am}
\begin{itemize}
    \item So with the data from "Global Health" I think it could be interesting to look into involvement of maternal mortality in different countries over time but also compare different European or African countries with each other  
    \item The data is quite clean so I don't need to clean it before I use it, which is very nice.
    \item This is just my first thoughts around my final project. I think I need to think a bit more about have to work with the data and which angle I can have on the data
\end{itemize}\bigskip


\section{21/11/2019}
\subsection{Thoughts/Intentions}
\textbf{Around 9:30 am}:  So I told my idea on my final project first to Petra and later Adela. While talking to Petra my original idea developed a bit, so now I'm making a comparison between Sweden, Denmark and Norway around maternal mortality and also child mortality. I choose these countries because healthcare wise the three countries are roughly alike, but is there  still a different  between have many children and mothers, who dies through time. 

\begin{itemize}
    \item  Adela approved my idea, but told my that perhaps I needed all European countries instead of just the Nordic countries.  
\end{itemize}

\subsection{Action}
\begin{itemize}
    \item Like yesterday I went to the "Our World in Data" website to find my new data on child mortality. 
    \item I went to the website. 
    \item Clicked on the category "Global Health" 
    \item  Found the subsection called "Child mortality" and downloaded the csv-file. 
\end{itemize}

\subsection{Final Thoughts}
\textbf{13.30 pm} Like yesterday I had a little look on the data on child mortality and once again the data is quite clean, which is nice. 
\begin{itemize}
    \item But I can see in the different files that there is quite a different time frame on how long the three countries have written their data down. Will this be a problem? 
\end{itemize}

\textbf{09:20am}
\begin{itemize}
    \item So with the data from "Global Health" I think it could be interesting to look into involvement of maternal mortality in different countries over time but also compare different European or African countries with each other  
    \item The data is quite clean so I don't need to clean it before I use it, which is very nice.
    \item This is just my first thoughts around my final project. I think I need to think a bit more about have to work with the data and which angle I can have on the data
\end{itemize}\bigskip


\section{26/11/2019}
\subsection{Thoughts/Intentions}
\begin{itemize}
    \item \textbf{15:32 pm}:  I have quite a bit of homework for Thursdays lesson and I need to do the following things: 
    \item Review the lessons 1, 2 and 3 in R for Social Scientist
    \item Do all the Exercises in lessons 2-3 and document them in your journal 
    \item Install tidyverse in R 
    \item Download and put the SAFI clean.csv somewhere on my computer where I can easily present it to R. 
    \item Look at lesson 4 and 5 in advance
\end{itemize}

\subsection{Action}
\textbf{Tidyverse}
\begin{itemize}
    \item  The first thing I'll do is to install Tidyverse. I need to run the install.packages("tidyverse") and command. 
    \item  But it seems like I can find my R-files from last week. Can I just install the program in R - I will try. 
    \item I just typed the install.packages("tidyverse") in R and it began to work, so I think I did it right! 
\end{itemize}\bigskip

\textbf{SAFI-clean.csv}
\begin{itemize}
    \item The next thing I will do is to download the SAFI file 
    \item So I will find the lesson plan and clicked on Week 4 to go to the "R for Social Scientists" guide 
    \item  At first I could not find the file, but then i reread Adela's post on Blackboard and I found the file on followed link: https://datacarpentry.org/r-socialsci/guide/ 
    \item Then I downloaded the file on to my computer and put it on my desktop, so I easily can find it for Thursdays lesson 
\end{itemize}\bigskip

\textbf{R for Social Scientist}
\begin{itemize}
    \item The next step I will do is to review the lessons 1, 2 and 3 for R in "R for Social Scientist"
    \item After I reviewed the lessons 1, 2 and 3 I need to do all the exercises for lesson 2 and 3
    \item The last thing I need to do for tomorrows lesson is to look a lesson 4 and 5 in advance  
\end{itemize}

\subsection{Final Thoughts}

\textbf{Around 11:35 on 27/11-2019 } I have done almost all the homework for this tomorrows lesson now. 

\begin{itemize}
    \item I was a bit difficult to find the SAFI file but I found it at lasted 
    \item I also had a bit of problems with the exercises. I could not quite finder out how to put it in to R and I got a lot of faults 
    \item I was quite confused while I read the lesson 4 and 5. I think it will make more sense, when Adela talks about it on tomorrows lesson. 
    \item I think the things from lesson 5 will be very relevant for my final project. It shows how to make graphs and different figure son data, which is what I'm exactly going for. 
\end{itemize}\bigskip


\section{28/11/2019}
\subsection{Thoughts/Intentions}
\textbf{11:15 am}:  So this morning I book a help time with Petra for Thursday next week at 11:30 am 
\begin{itemize}
    \item In all honesty I booked the time, so I can start on the project already now although my other exam is taking all my time at the moment, but I think it will help me in the long run.
    \item Another thing I need to do today is the git exercise, where I need to make my journal in Overleaf public 
\end{itemize}

\subsection{Action}
\textbf{Git exercise}
\begin{itemize}
    \item The first thing I need to do is to read about Git and the common commands in Week 3 slides and after that I need to open git bash and register with my username and email. 
    \item 
\end{itemize}


\section{02/11/2019}
\subsection{Thoughts/Intentions}
\textbf{09:15 am}:  My plan for this morning is to begin on my final project in R. 
\begin{itemize}
    \item I think the things from last week lesson will made sense in my final project because of the plots and graphs you can made in R. 
\end{itemize}

\subsection{Action}
\begin{itemize}
    \item The first thing I will do is to made a new script in R called my final project. I am still insecure about the final title for the project. 
    \item At the same time I will open the R file, which I made for the lesson on Thursday, so I can get some inspiration and general a idea on how to begin my final project. 
    \item I also need to open the "R for Social Scientists". 
    \item Back in my new R script the first thing I will do is to run the package "Tidyverse". I already installed the package last week, so the only thing I need to go is to run the command "library", which reinstall the package. 
    \item Afterwards I will upload my data sets to the script. The first one I will upload is the Maternal Mortality and secondly the data on child mortality. 
     \item With the data for child mortality I could not upload the file maybe because it is call "Interview" like Maternal mortality, so the next thing I will try is to call it "Interviews1", but it is still not working, so I decide to just work on the Maternal Mortality at first. 
     \item The next thing I will do is to put pipes on my data for maternal mortality because I only want to focus on Denmark, Norway and Sweden. The first country I will try on is Denmark. 
     \item I have some problems with which values I need to put in "filter" and "select". The things I would like to how in my final plot is country, year and maternal mortality ratio, but at the moment I can't figure out where to put the different values. I have tried all of them on different places. 
\end{itemize}
\subsection{Final Thoughts}
\textbf{Around 13:15 pm} So I have tried to begin on my final project today, but at the moment I have a lot of things, which doesn't work   
\begin{itemize}
    \item For my appointment with Petra on Thursday I need to ask her on how I upload the data for Child mortality in the same script as the other data 
    \item I also need to ask her on I'm need to mutate my data like on last week lesson
    \item I need to ask about, which values I need to have in "filter" and "select" to made it work. 
\end{itemize}\bigskip


\section{05/12/2019}
\subsection{Thoughts/Intentions}
\textbf{11:15 am}:  So today at 11:30 I had a appointment with Petra
\begin{itemize}
    \item Before my appointment with Petra I spent Monday trying to begin on my project. 
    \item On Monday I managed to upload my data for Maternal mortality, but I had a lot of problems on how to upload my data for child mortality and which values I needed to have in "filter" and "select"
\end{itemize}

\subsection{Action}
\begin{itemize}
    \item In my appointment with Petra today I asked her on the things I had troubles with on Monday. And she managed to give me some good answers on my questions, but she also helped me on how I changed the name on one of my columns in my data set because their where quite long and a bit hard to work with. 
    \item After my appointment with Petra I went to the library to continue my work with the project. 
    \item The first thing I did was to upload my other data on child mortality, which Petra told my I could just call whatever I liked, but I just called it "Interviews1". I already tried this on Monday, but at that time it did not work for me.
    \item After that I changed the name in one of the columns in child mortality to something shorter because the original title was quite long, which made it very difficult to work with. I also changed one of the columns in Maternal mortality to something shorter. 
    \item After that I looked at the pipes I made on Monday because Petra told me that I needed to change the "Interviews-Plotting" to "Interviews-Denmark" and in "filter" I needed to put "Denmark and in "Select" I needed to put "Entity, Year and Maternal Mortality" and subsequently it worked. 
    \item The next thing I tried was to made a plot for Denmark. I put year as X and Maternal mortality as Y in my plot. At the same thing I changed the color on the dots to yellow and I got a very fine graph for Danish maternal mortality. 
    \item After it all worked out with Denmark I once again made the pipes and the plots with maternal mortality for Norway and Sweden. It also worked, so now I got three very nice plots. 
    \item In the end I did all the steps once again, but this time it was with child mortality and again I did it with Denmark, Norway and Sweden. 
\end{itemize}
\subsection{Final Thoughts}
\textbf{Around 14:30 pm} So today I did quite a lot of work on my finale project and I am very satisfied with what I have achieved, but I will still like to make some changes.  
\begin{itemize}
    \item It really helped to get some helped from Petra. 
    \item Even through I have made some very fine plots I would like to made some graphs where both Denmark, Norway and Sweden are in because I think it will made it easier to compare the countries. 
\end{itemize}\bigskip


\section{06/12/2019}
\subsection{Thoughts/Intentions}
\textbf{08:30 am}:  Today I will try to continue on my work from yesterday and I will try to made some other graphs where, like I mentioned, both maternal - and child mortality for Denmark, Norway and Sweden are in. 
\begin{itemize}
    \item I think it will be quite hard to figure out in R, but I will try and if it doesn't work I will just made a comparison between the countries in their only plots.  
\end{itemize}

\subsection{Action}
\begin{itemize}
    \item Once again I will open the "R for Social Scientists" website because it is quite useful and of course I will open my final project in R. 
    \item So even though I just said that my goal for today was to figure out how to merge the three countries into one graphs I just remember that last week we learned to modify the plot to extract more information from it. So I just add transparency to avoid over plotting because many of the dots in my plots lies above each other.  
    \item It helped a bit but the dots for Danish and Swedish maternal mortality are still above each other. 
     \item The alpha function I just used for my plots only helped a little bit with the over plotting problem. So I will try the "geom-jitter" function, which will introduce a little bit of randomness into the position of my points. It function helped a bit on the points in the plot for Norway, which have been quite even in the last couple of plots. 
     \item And I will do the same with the plots for child mortality
    \item I have tried to make a bar plot for maternal mortality in Denmark, but it did not work. In the bar plots there are no y-axis, so the plot only shows the x-axis which is years and then the program just fills the whole plot up, which is not correct! 
    \item In continuation of the bar plots I also tried to put maternal mortality on the x-axis, but the program just made a "ERROR". 
    \item In my attempt to combine the data from Denmark, Norway and Sweden I have also tried to put Denmark and Norway in the same plot. As before I used the function "ggplot(data = Interviews-Denmark, aes(x = Year, y = Maternalmortality)) + geom-point()" and then I tried to put "Interviews-Norway" after "Interviews-Denmark" but it made a "ERROR". 
    \item I think the next thing I will try is to google how to combine plots in R. I found a website and tried some of the functions, but in R it just made a "ERROR". 
\end{itemize}
\subsection{Final Thoughts}
\textbf{10:50 am} Today I made some adjustments on the plots I made yesterday both on the plots for maternal - and child mortality  
\begin{itemize}
    \item I had some problems with different things, but I think I can solve them and if not I will just use the plots I already made because I think there are fine for what I am aiming to do with my final project. 
    \item But allover I am at good spirit!  
\end{itemize}\bigskip

\end{document}





